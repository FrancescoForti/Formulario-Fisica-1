\documentclass[10pt]{article}

\usepackage[italian]{babel}
\usepackage[applemac]{inputenc} %per accenti più facili su Mac
\usepackage[left=.9cm,right=.9cm,top=1.0cm,bottom=1.5cm,a4paper]{geometry}
\usepackage{multicol}
\usepackage{sectsty}
\usepackage{amsmath}
\usepackage[compact]{titlesec}


\newcommand{\versione}{Versione 2}
\newcommand{\data}{ 13 giugno 2011}

\setlength{\parindent}{5pt} 
\titlespacing{\subsection}{0pt}{3ex}{0ex}


\begin{document}



\chapterfont{\fontshape{ol}\fontseries{bc}\fontsize{14pt}{3pt}\selectfont}
\sectionfont{\fontshape{ol}\fontseries{bc}\fontsize{12pt}{2pt}\selectfont\bfseries}
\subsectionfont{\fontshape{ol}\fontseries{bc}\fontsize{11pt}{1pt}\selectfont\bfseries\itshape}
\columnseprule=1pt

\begin{center}\bfseries
	\Large Formulario di Fisica Generale I 
	
\end{center}

\begin{multicols}{3}

%inizio corpo del formulario

\subsection*{Cinematica}

Velocit\`a: $\vec v = \frac {d\vec r} {dt}$

Accelerazione: $\vec a = \frac {d\vec v} {dt} = \frac {d^2\vec r} {dt^2}$

\textbf{Moto uniformemente accelerato}

$  v - v_{0} = a \cdot t $

$  x - x_0 = v_0\cdot t + \frac 1 2 a t^2 $

$  x - x_0 =  \frac 1 2 ({v_0+v_x}) t $

$  v_x^2 -  v_0^2 = 2 a (x-x_0) $

Corpo in caduta da fermo:

$ v = \sqrt{2gh} $

$ t = \sqrt{2h/g} $

\textbf{Moto del Proiettile}

$\displaystyle y = x\cdot \tan\theta - \frac {g} {2 v_0^2 \cos^2 \theta} x^2$

$\displaystyle h_{max} = \frac {v_0^2 \sin^2\theta} {2 g} $

$\displaystyle x_{max} = \frac {v^2_0 \sin(2\theta)} g $

\textbf{Moto Circolare}

Velocit\`a angolare: $\omega = \frac {d\theta} {dt} $

Accel. angolare: $\alpha = \frac {d\omega} {dt} = \frac{d^2\theta}{dt^2}$

\textbf{Moto Circolare Uniforme}

$  \omega =  {2\pi} / {T} $

$ v_{\mathrm {tangenziale}} = \omega r $

$ a_{\mathrm {centripeta}} =  {v^2} / r  = \omega^2 r  $

\textbf{Moto Circolare Unif. Accel.}

$  \omega - \omega_{0} = \alpha \cdot t $

$  \theta - \theta_0 = \omega_0\cdot t + \frac 1 2 \alpha t^2 $

\textbf{Moto curvilineo}

$\vec a = a_T\hat\theta+a_R\hat r = \dfrac {d\left| \vec v \right|} {dt} \hat\theta -  \dfrac {v^2} {r} \hat r $

\subsection*{Sistemi a pi\`u corpi}

Massa totale: $m_{T} = \sum m_i = \int dm$

Centro di massa:

$\vec r_{CM} =  ( {\sum m_i \vec r_i} ) / m_T = (\int \vec r_i dm ) / m_T$

$\vec v_{CM} = {d\vec r_{CM}} / {dt} = {\sum m_i \vec v_i} / m_{T} $

$\vec a_{CM} = {d\vec v_{CM}} / {dt} = {d^2\vec r_{CM}} / {dt^2}  $

Momento di inerzia:

$I_\mathrm{asse} = \sum m_i r_i^2 = \int r^2 dm$

Teorema assi paralleli:

$I_\mathrm{asse} = I_\mathrm{CM} + m D^2 $

\subsection*{Forze, Lavoro ed Energia}

Legge di Newton: $\vec F = m \vec a$

Momento della forza: $\vec \tau = \vec r \times \vec F$

\textbf{Forze Fondamentali}

Forza peso:   $ F_g  = m g $

Forza elastica: $  F_{el} = - k (x -l_0)$

Gravit\`a: $\displaystyle \vec F_g = -G \frac {Mm} {r^2} \hat r$

Elettrostatica: $\displaystyle \vec F_E = \frac 1 {4\pi \varepsilon_0}
\frac {q_1 q_2} {r^2} \hat r$

\textbf{Forze di Attrito}

Statico: $| \vec F_S | \leq \mu_S | \vec N |$

Dinamico: $\vec F_D = -\mu_D | \vec N | \hat v$

Viscoso: $\vec F_V = -\beta \vec v$

\textbf{Lavoro }

$ L = \int_{x_i}^{x_f} \vec F \cdot d\vec l = \int_{\theta_i}^{\theta_f} \tau  d\omega $

Forza costante: $L = \vec F\cdot \vec l$

Forza elastica:

$ L=-\frac 1 2 k \left( x_f - l_0 \right)^2+\frac 1 2 k \left( x_i-l_0 \right)^2$

Forza peso: $L=-mgh$

Gravit\`a: $\displaystyle L = Gm_1m_2 \cdot \left( \frac 1 {r_f} - \frac 1 {r_i} \right)$

Elettrostatica: $\displaystyle L = \frac {q_1 q_2} {4\pi \varepsilon_0}
\cdot \left( \frac 1 {r_i} - \frac 1 {r_f} \right)$

Potenza: $\displaystyle P = \frac {dL} {dt} = \vec F \cdot \vec v = \tau \omega$

\textbf{Energia}

Cinetica: $K = \frac 1 2 m v^2 $

Rotazione: $ K = \left\{ \begin{matrix}
 \frac 1 2 m_T v_\mathrm{CM}^2 + \frac 1 2 I_\mathrm{CM} \omega^2  \\
 \frac 1 2 I_\mathrm{Asse Fisso} \omega^2   \\
 \end{matrix}
 \right.
 $

Forze vive: $K_f - K_i = L_{\mathrm {TOT}}$

Potenziale: $  U = -L = -\int_{x_i}^{x_f} \vec F \cdot d\vec l$

Meccanica: $ E = K + U = \frac 1 2 m v^2 + U $

Conservazione: $E_f - E_i = L_{\mathrm{NON\ CONS}}$

En. potenziale forze fondamentali:

Forza peso: $U(h)=mgh$

Forza elastica: $U(x) = \frac 1 2 k (x-l_0)^2$

Gravit\`a: $\displaystyle U(r) = -G \frac {m_1m_2} {r}$

Elettrostatica: $\displaystyle U(r) = \frac 1 {4\pi \varepsilon_0} \cdot \frac {q_1 q_2} {r}$

\subsection*{Impulso e Momento Angolare}

Quantit\`a di moto: $ \vec p = m\vec v$

Impulso: $\vec I = \vec p_f - \vec p_i = \int_{t_1}^{t_2} \vec F dt $

Momento angolare: $\vec L = \vec r \times \vec p$

Intorno ad un asse fisso: $|\vec L| = I_\mathrm{asse} \cdot \omega$

\textbf{Equazioni cardinali}

$\vec p_T = \sum \vec p_i = m_T \cdot \vec v_{CM}$

$\vec L_T = \sum \vec L_i = I_\mathrm{asse}\cdot \vec \omega$

I card: $\sum \vec F_\mathrm{ext} = d\vec p_T / dt = m_T \cdot a_\mathrm{CM}$

II card: $\sum \vec \tau_\mathrm{ext} = d\vec L_T / dt$

Asse fisso: $| \sum \vec \tau_\mathrm{ext} | = I_\mathrm{asse} \cdot \alpha_\mathrm{asse}$

\subsection*{Leggi di conservazione}

$\vec p_T = \mathrm{costante} \Leftrightarrow  \sum \vec F_\mathrm{ext} = 0 $

$\vec L_T = \mathrm{costante} \Leftrightarrow  \sum \vec \tau_\mathrm{ext} = 0 $

$ E = \mathrm{costante} \Leftrightarrow  L_\mathrm{NON CONS} = 0 $


\subsection*{Urti}

Per due masse isolate $\vec p_T = \mathrm{costante}$:

Anelastico: $v_f = \frac{m_1v_1+m_2v_2}{ m_1+m_2 }$

Elastico (conservazione energia):

$ \left\{
  \begin{matrix}
    m_1v_{1i} + m_2v_{2i} &=& m_1v_{1f}+m_2 v_{2f} \\
    m_1 ( v_{1i}^2-v_{1f}^2 ) &=& m_2 ( v_{2f}^2 - v_{2i}^2)
  \end{matrix}
\right. $

$\left\{ 
\begin{matrix}
  v_{1f} =& \frac {m_1-m_2} {m_1 + m_2} v_{1i} + \frac {2m_2} {m_1+m_2} v_{2i} \\
 v_{2f} =& \frac {m_2-m_1} {m_1 + m_2} v_{2i} + \frac {2m_1} {m_1+m_2} v_{1i}
\end{matrix}
\right. $


\subsection*{Moto Armonico}

$ x(t) = \: A \cos\bigl( \omega t + \phi_0 \bigr) $

$ v(t) = -\omega A \sin\bigl( \omega t + \phi_0 \bigr) $

$ a(t) = -\omega^2A \cos \bigl( \omega t + \phi_0 \bigr) = -\omega^2 x(t)$

$\displaystyle A  = \sqrt{x_0^2 + \left( \frac {v_0} \omega \right)^2 }$

$\displaystyle \phi_0 = \arctan \left( - \frac {v_0} {\omega x_0} \right)$

$f =  \omega / {2\pi}$, $T = {2\pi} / \omega$

Molla: $\omega = \sqrt{k / m}$

Pendolo: $\omega = \sqrt{g / L }$

\subsection*{Momenti di inerzia notevoli}

Anello intorno asse: $I= m r^2$

Cilindro pieno intorno asse: $I= \frac 1 2 m r^2$

Sbarretta sottile, asse CM: $I= \frac 1 {12} m L^2$

Sfera piena, asse CM: $I = \frac 2 {5} m r^2$

Lastra quadrata, asse $\perp$: $I = \frac 1 {6} m L^2$

\subsection*{Gravitazione}
3$^a$ legge di Keplero: $T^2=\left(\frac{4\pi^2}{G M_S}\right)R^3$

Vel. di fuga: $v=\sqrt{\frac{2G M_T}{R_T}} $

\subsection*{Elasticit\`a}
Modulo di Young: $F/A = Y \cdot \Delta L/L$

Compressibilit\`a: $\Delta p = - B \cdot \Delta V/V$

Modulo a taglio: $F/A = M_t \cdot \Delta x/h$

\subsection*{Fluidi}
Spinta di Archimede $B_A=\rho_L V g$

Continuit\`a: $A \cdot v = \mathrm{costante}$

Bernoulli: $p + \frac 1 2 \rho v^2 + \rho g y =  \mathrm{costante}$

\subsection*{Onde}
Velocit\`a $v$, pulsazione $\omega$, lunghezza d'onda $\lambda$, periodo $T$, frequenza $f$, numero d'onda $k$.

$v=\omega/k=\lambda/T=\lambda f$

$\omega = 2\pi/T, \quad k=2\pi/\lambda$

\textbf{Onde su una corda}

Velocit\`a: $v=\sqrt{T/\mu}$

Spostamento: $y=y_\mathrm{max}\sin(kx - \omega t)$

Potenza: $P=\frac 1 2 \mu v (\omega y_\mathrm{max})^2$

\textbf{Onde sonore}

Velocit\`a: $v=\sqrt{B/\rho} = \sqrt{\gamma p/ \rho} $ 

$v(T) = v(T_0)\sqrt{T/T_0}$

Spostamento: $s=s_\mathrm{max}\cos(kx - \omega t )$ 

Pressione: $\Delta P=\Delta P_\mathrm{max}\sin(kx - \omega t)$

$\Delta P_\mathrm{max}=\rho v \omega s_\mathrm{max}$

Intensit\`a: $I=\frac 1 2 \rho v (\omega s_\mathrm{max}) ^2 =
 \frac{\Delta P_\mathrm{max}^2}{2 \rho v}$

Intensit\`a(dB): $\beta = 10\log_{10}\frac{I}{I_0}$

Soglia udibile $I_0=1.0\times 10^{-12}\,\mathrm{W/m}^2$

\textbf{Effetto Doppler}

$\displaystyle  f^\prime = \left( \frac{v+v_O\cos\theta_O}{v-v_S\cos\theta_S} \right) f$

\pagebreak
%+++++++++++++++++++++++++++++++++++++
\subsection*{Termodinamica}

%Teoria cinetica

\textbf{Primo principio}

Calore e cap. termica: $Q = C\cdot\Delta T$

Calore latente di trasf.: $L_t = Q/m$

Lavoro \underline{sul} sistema: $dW = - p dV$

En. interna: $\Delta U = 
\left\{ \begin{matrix}
Q + W_\mathrm{sul sistema} \\
Q - W_\mathrm{del sistema}
 \end{matrix}\right.
$

Entropia: $\Delta S_{AB} = \displaystyle\int_{A}^{B} \frac{dQ_{REV}}{T} $

\textbf{Calore specifico}

Per unit\`a di massa: $c = C/m$

Per mole: $c_m = C/n$

Per i solidi: $c_m \approx 3R$

Gas perfetto: $c_p - c_V = R$

\begin{tabular}{c |c c c}
 & $c_V$ &$ c_p $& $ \gamma = c_p / c_V$ \\[2pt]
monoatom. & $\frac{3}{2}R$ & $\frac{5}{2}R$ & $\frac{5}{3}$ \\[2pt]
biatomico & $\frac{5}{2}R$ & $\frac{7}{2}R$ & $\frac{7}{5}$\\
\end{tabular}


\textbf{Gas perfetti}

Eq. stato: $pV = nRT = N k_b T$ 

%Energia interna: $U = \frac{1}{2} n R T \times \mathrm{gradi} $ %di libertˆ?

Energia interna: $\Delta U = n c_V \Delta T $

Entropia: $\Delta S = n c_V \ln{\frac{T_f}{T_i}} + n R \ln{\frac{V_f}{V_i}} $

%\textbf{Trasformazioni quasi statiche}

\underline{Isocora} ($\Delta V = 0$):

$W = 0$ ; $Q = n c_v \Delta T$

\underline{Isobara} ($\Delta p = 0$):

$W = - p \Delta V$ ;  $Q = n  c_p \Delta T$

\underline{Isoterma} ($\Delta T = 0$):

$W = -Q = - nRT \ln{\frac{V_f}{V_i}}$

\underline{Adiabatica} ($Q = 0$):  $pV^{\gamma} = $ cost. 

$TV^{\gamma - 1} = $ cost. ; $p^{1-\gamma} T^\gamma =$ cost. 

$W = \Delta U = \frac{1}{\gamma - 1} (P_f V_f - P_i V_i)$


\textbf{Macchine termiche}

Efficienza: $\eta = \frac{W}{Q_H} = 1 - \frac{Q_C}{Q_H}$

C.O.P. frigorifero $ = \frac{Q_C}{W}$

C.O.P. pompa di calore$ = \frac{Q_H}{W}$

Eff.  di Carnot: $\eta_{REV} = 1 - \frac{T_C}{T_H}$

Teorema di Carnot: $\eta \leq \eta_{REV}$ 


\textbf{Espansione termica dei solidi}

Esp. lineare: $\Delta L / L_i = \alpha \Delta T$

Esp. volumica: $\Delta V / V_i = \beta \Delta T$

Coefficienti: $\beta = 3 \alpha$

$\beta$ gas perfetto, $p$ costante: $\beta = 1/T$

\textbf{Conduzione e irraggiamento}

Corrente termica: 

$\mathcal{P} = \frac{\Delta Q}{\Delta t} = \frac{\Delta T}{R} = \frac{kA}{\Delta x}\Delta T$ 

Resistenza termica: $R = \frac{\Delta x}{k A}$

Resistenza serie: $R_{eq} = R_1 + R_2$

Resistenza parallelo: $\frac{1}{R_{eq}} = \frac{1}{R_1} + \frac{1}{R_2}$

Legge Stefan-Boltzmann: $\mathcal{P} = e \sigma A T^{4}$

L. onda emissione: $\lambda_{max} = \frac{2.898 \,\mathrm{mm K}}{T}$

\textbf{Gas reali}

Eq. Van Der Waals:

 $(p + a (\frac{n}{V})^{2})(V - nb) = nRT$

\subsection*{Calcolo vettoriale}

Prodotto scalare: 

$\vec A \cdot \vec B = | \vec A || \vec B |\cos\theta $

$\vec A \cdot \vec B = A_xB_x + A_yB_y + A_zB_z$

$|\vec A| = \sqrt{\vec A \cdot \vec A} = \sqrt{A_x^2 + A_y^2 + A_z^2}$

versore: $\hat A =  {\vec A} /  {|\vec A|}$

Prodotto vettoriale:

$\vec A \times \vec B = \begin{vmatrix}
\hat i  &  \hat j  & \hat k \\
A_x     &   A_y    &  A_z \\
B_x     &   B_y    &  B_z
\end{vmatrix}$

$ \begin{matrix}
\vec A \times \vec B  & = & (A_yB_z - A_zB_y)\hat i  \\
 & + & (A_zB_x - A_xB_z)\hat j  \\
 & + & (A_xB_y - A_yB_x) \hat k
\end{matrix}$

\setlength{\parindent}{0pt} 

\subsection*{Costanti fisiche}

\textbf{Costanti fondamentali}

Grav.: $G = 6.67\times 10^{-11}\,\mathrm{m^3/(s^2\cdot kg)}$\\
Vel. luce nel vuoto: $c = 3.00\times 10^8\,\mathrm{m/s}$\\
Carica elementare: $e = 1.60\times 10^{-19}\,\mathrm{C}$\\
Massa elettrone: $m_e = 9.11\times 10^{-31}\,\mathrm{kg}$\\
Massa protone: $m_p = 1.67\times 10^{-27}\,\mathrm{kg}$\\
Cost. dielettrica: $\varepsilon_0 = 8.85\times 10^{-12}\,\mathrm{F/m}$\\
Perm. magnetica:  $\mu_0 = 4\pi\times 10^{-7}\,\mathrm{H/m}$\\
Cost. Boltzmann: $k_b = 1.38 \times 10^{-23}\,\mathrm{J/K}$\\
N. Avogadro: $N_A = 6.022 \times 10^{23}\,\mathrm{mol^{-1}}$\\
C. dei gas: $ R = \left\{ \begin{matrix} 8.314 \,\mathrm{J / (mol \cdot K)} \\
0.082 \,\mathrm{L\cdot atm / (mol \cdot K)} \end{matrix}\right.$\\
C. Stefan-Boltzmann: \\ 
\phantom{aaaaaaa}    $  \sigma = 5.6 \times 10^{-8} \,\mathrm{W/(m^2 \cdot K^4)}$\\

\textbf{Altre costanti}

Accel gravit\`a sulla terra: $g = 9.81\,\mathrm{m/s^2}$\\
Raggio terra:  $R_T = 6.37\times 10^{6}\,\mathrm{m}$\\
Massa terra:  $M_T = 5.98\times 10^{24}\,\mathrm{kg}$\\
Massa sole:  $M_S = 1.99 \times 10^{30}\,\mathrm{kg}$\\
Massa luna:  $M_L = 7.36 \times 10^{22}\,\mathrm{kg}$\\
Vol. 1 mole di gas STP: $V_{STP} = 22.4 \,\mathrm{L}$\\
Cal. specifico acqua: $4186 \,\mathrm{J/(kg \cdot K)}$\\
Temp 0 assoluto $\theta_0 = -273.15\,\mathrm{^\circ C}$

\subsection*{Trigonometria}

$\sin^2(\alpha)+\cos^2(\alpha) = 1, \tan(\alpha)=\frac{\sin(\alpha)}{\cos(\alpha)}$ \\
$\sin(-\alpha) = -\sin(\alpha)$, $\cos(-\alpha) = \cos(\alpha)$\\
$\sin(\alpha \pm \beta) = \sin(\alpha)\cos(\beta) \pm
\cos(\alpha)\sin(\beta)$\\
$\cos(\alpha \pm \beta) = \cos(\alpha)\cos(\beta) \mp
\sin(\alpha)\sin(\beta)$ \\
$\sin(\alpha) = \pm\cos(\pi/2\mp\alpha) = \pm\sin(\pi\mp\alpha) $\\
$\cos(\alpha) = \sin(\pi/2\pm\alpha) = -\cos(\pi\pm\alpha)$\\
$\sin^2(\alpha) = \frac{1-\cos(2\alpha)}{2}$,
$\cos^2(\alpha) = \frac{1+\cos(2\alpha)}{2}$\\
$\sin(\alpha) + \sin(\beta) = 2 \cos\frac{\alpha-\beta}{2} \sin\frac{\alpha+\beta}{2} $\\
$\cos(\alpha) + \cos(\beta) = 2 \cos\frac{\alpha-\beta}{2} \cos\frac{\alpha+\beta}{2} $

\subsection*{Derivate}

$ \frac {d} {dx} f(x)= f^\prime(x) $

$\frac {d} {dx}(a\cdot x) = a f^\prime(a\cdot x)$

$ \frac {d} {dx} f(g(x)) = f^\prime(g(x)) \cdot g^\prime(x)  $

$\frac d {dx} x^n = n x^{n-1}$

$\frac d {dx} \frac 1 {x^n} = -n \frac 1 {x^{n+1}}$

$\frac d {dx} e^x = e^x$

$\frac d {dx} \ln x = \frac 1 x$

$\frac d {dx} \sin(x) = \cos (x)$

$\frac d {dx} \cos(x) = -\sin(x)$

\subsection*{Integrali}

$\displaystyle\int f(x) dx = I(x)$

$\displaystyle\int f(x-a) dx = I(x-a)$

$\displaystyle\int f(a\cdot x) dx = \frac {I(a\cdot x)} {a}$

$\displaystyle \int x^n dx = \frac {x^{n+1}} {n+1},\:n\neq -1$

$\displaystyle\int \frac 1 {x^n} = - \frac 1 {(n-1)} \cdot\frac 1 { x^{n-1}},\: n\neq 1$

$\displaystyle\int \frac 1 x dx = \ln x$

$\displaystyle\int e^x dx = e^x$

$\displaystyle\int \sin(x) dx = \cos(x)$

$\displaystyle \int \cos(x) dx = -\sin(x)$

$\displaystyle \int_{x_0}^{x_1} f(x) dx = I(x_1) - I(x_0)$

\subsection*{Approssimazioni $(x_0 = 0)$}

$\sin{x} = x + O(x^2)$

$(1+x)^{\alpha} = 1 + \alpha x + O(x^2)$

$\ln(1+x) = x + O(x^2)$


%fine corpo formulario


\end{multicols}

\vfill

{\tt Francesco.Forti@pi.infn.it et al.}\hfill  \versione , \data.
\end{document}
